\documentclass[12pt, letterpaper]{article}
\usepackage{tikz}
\usetikzlibrary{positioning}
\usepackage{graphicx}
\usepackage{flafter} % needed for figures
\usepackage{hyperref} % needed for a clickable table of contents
\usepackage[export]{adjustbox} % needed for adjustable images
\usepackage{polyglossia} % needed for catalan support
\setmainlanguage{catalan}
\graphicspath{ {images/} }

% rename for quotes
\let\oldquote\quote
\let\endoldquote\endquote
\renewenvironment{quote}[2][]
  {\if\relax\detokenize{#1}\relax
     \def\quoteauthor{#2}%
   \else
     \def\quoteauthor{#2~---~#1}%
   \fi
   \oldquote}
  {\par\nobreak\smallskip\hfill(\quoteauthor)%
   \endoldquote\addvspace{\bigskipamount}}

% information
\title{\textbf{Pràctica 1 || Aquæductus}}
\author{Pablo Fraile Alonso}
\date{\today}

% document
\begin{document}
% title
\maketitle
\thispagestyle{empty}
\newpage
\tableofcontents
\listoffigures
\newpage

% begin contents
\section{Funcionament algorisme}

% backtracking
\subsection{Primera idea: Backtracking}

\subsubsection{Per què no utilitzar Backtracking en aquest cas}
cost de $O(n!)$ .

% principi optimitat
\subsection{Alternativa a backtracking: Principi d'optimitat}
Abans de poder comentar la solució, hem d'entendre que és el principi d'optimitat: 
\begin{quote}{Richard E.Bellman} Principi d'optimitat: Una política òptima té la propietat que sigui quin sugui l'estat inicial i la decisió inicial, les decisions restants han de construir una política òptima respecte a l'estat resultat de la primera decisió.
\end{quote}

Per tant, seguint aquesta definició podem dir que un problema podrà ser resolt seguint el principi d'optimitat si la seva solució òptima pot ser construida eficientment a partir de les solucions òptimes dels seus subproblemes. En altres paraules, que podem resoldre un problema gran donades les solucions dels seus problemes petits.

\subsubsection{Aplicació i funcionament en el nostre cas d'ús}
\subsubsection{Demostració per reducció al absurd}
Donat un aqueducte que va d'un punt A a un punt J i que té recorregut R\textsubscript{a...j} és el óptim (figura: \ref{demostracio:atoj}).

\begin{figure}[hbt!]
\begin{center}
\begin{tikzpicture}[roundnode/.style={circle, draw=green!60, fill=green!5, very thick, minimum size=7mm},]
% nodes
\node[roundnode]      (A)                              {A};
\node[roundnode]      (J)       [right=3cm of A]           {J};
% lines
    \draw[->] (A.east) -- (J.west) node[midway, below] {R\textsubscript{a...j}};
\end{tikzpicture}
\caption{Aqueducte de punt A a punt J}
\label{demostracio:atoj}
\end{center}
\end{figure}

Ara assumirem que aquest recorregut passa per el punt K, per tant ara podem separar el recorregut com R\textsubscript{a...k} \& R\textsubscript{k...j} (FIGURA TAL) i que del punt A al punt K pot haver-hi un recorregut mes optim, que anomenarem R'\textsubscript{a...k} 

\begin{tikzpicture}[
roundnode/.style={circle, draw=green!60, fill=green!5, very thick, minimum size=7mm},
]
%Nodes
\node[roundnode]      (maintopic)                              {A};
\node[roundnode]      (rightsquare)      [right=of maintopic] {k};
\node[roundnode]      (rightsquare2)      [right=of rightsquare] {J};

%Lines
\draw[->] (maintopic.east) -- (rightsquare.west);
\draw[->] (rightsquare.east) -- (rightsquare2.west);
\end{tikzpicture}

Si R'\textsubscript{a...k} és més òptim que R\textsubscript{a...k}, llavors vol dir que:
R'\textsubscript{a...k} < R\textsubscript{a...k}.
Llavors:
R'\textsubscript{a...k} + R\textsubscript{k...j} < R\textsubscript{a...k} + R\textsubscript{k...j} 

Però aquesta afirmació NO pot ser certa! Ja que en un principi hem assegurat que R\textsubscript{a...k} + R\textsubscript{k...j} era la solució òptima i per tant no hi pot haver-hi cap més petita que aquesta.


\section{Cost algorisme}
\subsection{Iteratiu}
\subsection{Recursiu}
    
\section{Problemes/consideracions}
\subsection{Nombres en c++}

\section{Conclusions}

\end{document}
